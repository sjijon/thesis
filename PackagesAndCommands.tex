%% !TEX root = ThesisManuscript_SJ.tex
%%
%%
%%	PACKAGES
%%_______________________________________________
%%
%% 	Basic setup
%%
\usepackage[latin9]{inputenc}
\usepackage[T1]{fontenc}
\usepackage[francais,english]{babel}
\usepackage{amsmath,amssymb,amsthm}
%%
%% 	Format
%%
\usepackage[usenames,dvipsnames,table]{xcolor}
\usepackage[colorlinks=true,linkcolor=MyBlue,citecolor=MyBlue, urlcolor=MyBlue]{hyperref}
%\usepackage[colorlinks=true,linkcolor=black,citecolor=black, urlcolor=black]{hyperref}
\usepackage{mathrsfs}								% RSFS font
\usepackage{nicefrac}								% Nice fractions
\usepackage{setspace} 								%Interlinear space
\usepackage{footnote} 								% To format footnotes
%\counterwithout{footnote}{chapter}
\usepackage[width=0.9\textwidth,labelfont=bf,textfont=small]{caption}
	\captionsetup[table]{font={stretch=1.2}}
	\captionsetup[figure]{font={stretch=1.2}}
\usepackage{setspace}  \linespread{1.5}					% Interlinear space
\usepackage{lineno} 									% Line numbers

%%
%% Biblio layout
%%
%\usepackage[sort&compress,numbers,square]{natbib}		% 
\usepackage{natbib}
\setcitestyle{authoryear,open={(},close={)}}
%%
%% 	Elements
%%
\usepackage{graphicx} 								% To insert figures
\usepackage{float}
\usepackage{array} 									% Arrays
\usepackage{booktabs} 								% Nice tables
\usepackage{multirow}
\usepackage[final]{pdfpages} 							% To insert pdfs
\usepackage{appendix}
\usepackage{enumitem}
\usepackage{eurosym}
\usepackage{algorithm}
\usepackage{algpseudocode}
%%
%%
%%
%\usepackage{glossaries}
%%\usepackage[nonumberlist]{glossaries}
%\setacronymstyle{long-short}
%\loadglsentries{Abbreviations2}
%\makenoidxglossaries
%%
%% Pages layout
%%
\usepackage{fancyhdr} 
\setlength{\headheight}{26pt}
\pagestyle{fancy}
\renewcommand{\headrulewidth}{0.05pt}
\renewcommand{\footrulewidth}{0pt}
\fancyhf{}
\fancypagestyle{others}{
	\fancyfoot[LE,RO]{\textcolor{black}{\thepage}}
	\fancyhead[RE,RO]{\textcolor{black}{\small \leftmark}}
}
\fancypagestyle{none}{
	\fancyfoot[LE,RO]{}
	\fancyhead[RE,RO]{}
}
\fancypagestyle{chapters}{
	\fancyhead[LE,LO]{\small CHAPTER \thechapter}
	\fancyhead[RE,RO]{\textcolor{black}{\small \leftmark}}
}
\fancypagestyle{appendices}{
	\fancyhead[LE,LO]{\small APPENDIX \thesection}
	\fancyhead[RE,RO]{\textcolor{black}{\small \leftmark}}
}
\fancypagestyle{biblio}{
	\fancyhead[LE,LO]{}
	\fancyhead[RE,RO]{\textcolor{black}{\small \leftmark}}
}


\usepackage{afterpage}
\usepackage{lscape} 
%%
%% LOF layout
%%
\usepackage{chngcntr} 								% To number figures within sections
	\counterwithin{figure}{chapter}			
	\counterwithin{table}{chapter}

% VER LUEGO SI PUEDO ARREGLAR
%\usepackage{tocloft} %% It changes all the layout (notably the openright option). 
%\setlength{\cftfignumwidth}{32pt}

%%_______________________________________________
%%
%%	COMMANDS
%%_______________________________________________
%% 	Theorems
\newtheorem{prop}{Proposition}
\newtheorem{defi}{Definition}
\newtheorem{propi}{Property}
%%
%% 	Hyperrefs
%%
\renewcommand\thealgorithm{\thechapter.\arabic{algorithm}} 
\renewcommand\tablename{\textsc{Table}}
\newcommand{\algoref}[1]{\hyperref[#1]{pseudo-algorithm~\ref{#1}}}
\renewcommand{\eqref}[1]{\hyperref[#1]{eq.~(\ref{#1})}}
%\newcommand{\figref}[1]{\hyperref[#1]{Figure~\ref{#1}}}
\newcommand{\figref}[1]{\hyperref[#1]{fig.~\ref{#1}}}
\newcommand{\tablesref}[1]{\hyperref[#1]{tables~\ref{#1}}}
\newcommand{\chapref}[1]{\hyperref[#1]{chapter~\ref{#1}}}
\newcommand{\secref}[1]{\hyperref[#1]{section~\ref{#1}}}
\newcommand{\sectionsref}[1]{\hyperref[#1]{sections~\ref{#1}}}
\newcommand{\propref}[1]{\hyperref[#1]{proposition~\ref{#1}}}
\newcommand{\apxref}[1]{\hyperref[#1]{Appendix~\ref{#1}}}
\newcommand{\figuresref}[1]{\hyperref[#1]{figs.~\ref{#1}}}
\newcommand{\eqsref}[2]{\hyperref[#1]{eqs.~(\ref{#1})--(\ref{#2})}}

\def\sectionautorefname{Section}
\def\subsectionautorefname{Section}
\def\subsubsectionautorefname{Section}

\newcommand*{\nolink}[1]{\begin{NoHyper}#1\end{NoHyper}}
%%
%% ToC
%%
\setcounter{secnumdepth}{3} 	% To number subsubsections
\setcounter{tocdepth}{3} 		% To show the subsubsections in the ToC
%\renewcommand\thesubsubsection{\alph{subsubsection}.} 
%\renewcommand{\thechapter}{\Roman{chapter}} 
%%
%%_______________________________________________
%%
%%	MY COLORS
%%_______________________________________________
\definecolor{MyBlue}{RGB}{0,100,170}
\definecolor{MyDarkBlue}{rgb}{0, 0.25, 0.45}
\definecolor{MyBrown}{rgb}{0.28, 0.20, 0.20}
\definecolor{MyOrange}{RGB}{255,80,30}
\definecolor{MyOldOrange}{rgb}{0.75, 0.25, 0.0}
\definecolor{MyRed}{RGB}{200,0,0}
\definecolor{MyGray}{RGB}{200,200,200}
\definecolor{MyDarkGray}{rgb}{0.5,0.5,0.5}
\definecolor{MyGreen}{rgb}{0.33, 0.5, 0.18}
\definecolor{MyDarkGreen}{rgb}{0.15, 0.25, 0.18}
\definecolor{MyTurquoise}{rgb}{0, 0.4, 0.4}
\definecolor{MyViolet}{rgb}{0.44, 0.16, 0.39}
\definecolor{MyYellow}{rgb}{1, 0.65, 0}
%%_______________________________________________
%%
%% !!!! DELETE WHEN FINISHED
%%_______________________________________________
% 	Colored text
\newcommand{\rev}[1]{\textcolor{MyRed}{#1}}
\newcommand{\note}[1]{\textcolor{MyGray}{#1}}
% 	Labels
%\usepackage[right]{showlabels}	
