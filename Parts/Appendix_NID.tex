% !TEX root = ../ThesisManuscript_SJ.tex
%%
%%	INTERDISCIPLINARY NOTE - R�seau Doctorale EHESP
%%_______________________________________________
\section{Interdisciplinary note}
\label{NID}
\markboth{INTERDISCIPLINARY NOTE}{}

During my PhD, I was registered to the RDSP\footnote{R�seau doctoral en sant� publique.} (Public health doctoral network), which is organized by the EHESP\footnote{�cole des hautes �tudes en sant� publique.} (School of high studies in public health). As a part of the RDSP program, I wrote an interdisciplinary note that places my doctoral research within a broader, interdisciplinary context. An excerpt% The corresponding document, titled {\it \guillemotleft~Prevention of infectious diseases: from a game-theoretic approach to a multi-level, interdisciplinary perspective~\guillemotright}, is included in the following pages.

%%%
%%%________________________________________
%%%
%%% Interdisciplinary note
%\newpage
%\markboth{INTERDISCIPLINARY NOTE}{}
%\includepdf[pages = -,
%		frame = false,
%		link = true,
%		linkname = NID,
%		openright = false,
%		offset = 0cm -0.5cm,
%		pagecommand = { }]
%		{Parts/Documents/NID_EHESP/NID_EHESP.pdf}
%%		{DRAFT_FigsAndDocs/NID.pdf}


%%________________________________________________________________________
%%
%% 	2. The interdisciplinary context of the subject
%%________________________________________________________________________
\begin{center}
\large \bfseries
\guillemotleft The prevention of infectious diseases: \\an interdisciplinary challenge \guillemotright
\end{center}

Controlling an epidemic is a multi-level, interdisciplinary challenge. The decision-making on whether or not to use prevention is done at the individual level, whereas the preventive method implementation and its impact may be evaluated at the populational level. Public health authorities' recommendations rely on evidence and studies from the scientific community. Public policies are established under limited resources allocated through economical analyses. Finally, prevention promotion and prevention programmes require the active participation of the the media and the general population.

\subsubsection*{Public health authorities}
Public health authorities may facilitate access to information and epidemiological data about infectious diseases and the available preventive and therapeutic methods and programmes. In addition, public health authorities need to address the subject of prevention hesitancy~\cite[]{Larson2016} and difficulties in its accessibility, in order to improve prevention coverage. Reducing the perceived prevention-related barriers~\cite[]{Larson2014,Jarrett2015,Coleman2017} (which involve monetary and non-monetary aspects), may encourage individuals to adopt preventive methods against infectious diseases and thus, yielding broad prevention uptake. In addition, monetary and/or non-monetary incentives may be used as strategies to improve prevention uptake~\cite[]{Jarrett2015}.

\subsubsection*{The participation of the community}
The risk perception as well as the acceptability of prevention may vary between individuals~\cite[]{Larson2014,Blumenthal2019}. Hence, it is essential for individuals to have a fair perception of their risk of infection, to follow the public health authorities recommendations and reach the health facilities that offer prevention interventions that suit them the best.

Prevention programmes may be importantly reinforced by the participation and support of the community~\cite[]{ConcertationFR2016}. Unfortunately, the community could also play the role of a barrier against prevention efforts, in the cases where individuals may be susceptible to peer-pressure and social stigma (for instance, HIV-related stigma impacting PrEP acceptability~\cite[]{Young2014} and peer-pressure against vaccination~\cite[]{Larson2014}).

\subsubsection*{Economical decisions}
Prevention has shown to be cost-effective, with respect to treatment, for infectious diseases such as measles~\cite[]{Strebel2013}, HIV~\cite[]{Cambiano2018}, influenza~\cite[]{Dabestani2019} and other emerging infectious diseases~\cite[]{Heymanng2014}. These results place prevention as a public health priority regarding ressources allocation.

Affordability and accessibility being key aspects in prevention adoption, it is also important to highlight the necessity of reducing socio-economical inequalities. For instance, income has been identified as a factor impacting vaccination acceptability~\cite[]{Larson2014}, and gaps still remain in the path to HIV elimination at the global scale, given that only 1 out of 10 MSM has access to HIV prevention interventions~\cite[]{WHO_Gap}. %Integrating PrEP into routine preventive health (instead of targeted programmes) may help avoid exacerbating disparities~\cite[]{Calabrese2017}.

\subsubsection*{Mathematical epidemiology}
The mathematical modeling of infectious diseases has become a key tool for epidemiology, helping understanding epidemics' dynamics, estimating the values of parameters not directly measurable, predicting future states of the system and selecting optimal intervention designs~\cite[]{Brauer2017}.

Disease transmission has been mostly studied using compartimental models~\cite[]{Brauer2017,Rohani2008}, which allow to study the epidemic dynamics at the population level, by tracing the transition of individuals trough different states, regarding the infection. Recently, individual behavior and its impact on epidemic dynamics has also been introduced in the modeling of infectious diseases prevention~\cite[]{Verelst2016}. For instance, game-theoretic approaches have been used. Game theory allows, among other things, to model rational individual's decision-making and strategies. Hence, game-theoretic models allow to include strategic individual-level decision-making on whether or not to adopt a preventive method to avoid the infection.

Mathematical models are useful when they are adapted to the research question and its context. Therefore, modelers need to find the right balance between the complexity that helps increasing model accuracy, and the simplicity that allows a model to clearly be understood, to be flexible and to be solved without requiring high levels of computing power~\cite[]{Rohani2008}. In addition, models are parametrized using available data, which highlights the importance of good-quality of data collection.

\subsubsection*{Scientific communication}
The dissemination of information about epidemics and disease burden may shape individual's sense of their risk of infection, as well as provide awareness on the available preventive and therapeutic tools. It is thus essential to account for the impact of scientific communication within public prevention efforts. Unfortunately, there is evidence of misinterpretation of research results spreading through mass media~\cite[]{Haneef2015}.

For individuals to make well-informed decisions, it is important, on the one hand, that researchers communicate their results in the most efficient way possible so misinterpretations are reduced to a minimum; on the other hand, that media transmit scientific results accurately. For instance, explaining the meaning of prevention-parameters estimations, such as the preventive method effectiveness may help broaden prevention rollout programmes~\cite[]{Underhill2016}.
