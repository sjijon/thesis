% !TEX root = ../ThesisManuscript_SJ.tex
%%
%%	Behavioral models for COVID-19
%%_______________________________________________
\let\cleardoublepage\clearpage 
\renewcommand\setthesection{\Alph{section}} 
\counterwithin{figure}{section}

\chapter*{Appendices}
\addcontentsline{toc}{chapter}{Appendices}

\section{Behavioral epidemiology and the COVID-19 pandemic}
\label{BehavEpiCOVID}
\markboth{BEHAVIORAL EPIDEMIOLOGY AND COVID-19}{}

As of March 31, 2021, around 128 million cases of SARS-CoV-2\footnote{Severe acute respiratory syndrome coronavirus 2, the causative agent of COVID-19, the severe coronavirus disease. See the review by~\cite{Salzberger2020} for more details on the epidemiology of SARS-CoV-2.} infections were reported to the WHO worldwide, and resulted in $\sim2.8$ million deaths \cite[]{WHO_CovidDashboard}. Public health authorities have established a variety of mandatory non-pharmaceutical interventions aiming to control the still ongoing epidemic, successfully reducing the number of new infections worldwide \cite[]{Bo2021}. However, heavy interventions such as national-level lockdowns and closing schools and stores are not tenable in the long run, so the development of immunization tools has been of great interest. Since mid-December 2020, several vaccines were put on the market \cite[]{WHO_CovidVaccines}, and vaccination campaigns started around the world. As of March 31, 2021, $\sim338$ million people had received at least one dose of COVID-19 vaccine worldwide \cite[]{OWID_CovidVaccination}. The discussion on ending the COVID-19 epidemic at the European region started with a petition in Germany in January 2021 \cite[]{ZeroCovid_EU}, provoking debates in other European countries, such as France \cite[]{ZeroCovid_FR} and UK \cite[]{ZeroCovid_UK}. 

Mathematical models have assisted public-health decision-making by estimating disease parameters (such as the effective reproduction number and the duration of the different stages of disease progression), as well as predicting the impact of intervention measures on the COVID-19 pandemic \cite[]{Xiang2021}. In particular, behavioral epidemiology has been used to study the impact of epidemic control interventions requiring the voluntary participation of individuals, such as social distancing \cite[]{Gupta2020}, stay-at-home interventions \cite[]{Kabir2020} and vaccination \cite[]{Choi2020,Jentsch2021}\footnote{Last search of bibliography related to behavioral epidemiology and COVID-19: April 15th, 2021.}. Our approach may give insights into the subject of vaccination against COVID-19 and health programs aiming at epidemic elimination, by highlighting how essential an accurate perception of the infection risk and a reduction of the perceived cost of preventing SARS-CoV-2 infection are. 

Individuals' perception of the risk of SARS-CoV-2 infection has shown to be strongly related to their direct experience with the virus and the local epidemic situation \cite[]{Mansilla2020,Elharake2021}. In addition, a recent study found that individuals' knowledge and beliefs about the pandemic were associated with individuals' information sources, which were strongly determined by sociodemographic characteristics \cite[]{Ali2020}. Therefore, availability and broad access to accurate information would need to be ensured. Official, governmental sources may help ensuring individuals' trust in the information, which may translate in the successful adoption of preventive behaviors \cite[]{Lim2020}.

Attitudes towards vaccination against COVID-19 vary widely between countries. A survey about vaccine acceptability \cite{Wuoters2021} found that only 44\% of the French respondents would potentially get vaccinated, versus 81\% of the UK responders. The cost perceived for vaccination may be thus reduced by increasing transparent and accurate communication about vaccine safety and effectiveness, as well as ensuring broad availability and accessibility \cite[]{Wuoters2021}. 

%Testing, tracing and isolating infected individuals has become one of the main strategies to control the epidemic \cite[]{WHO_COVID19Strategy}. However, an additional issue of the COVID-19 epidemic is the high number of asymptomatic infections: an estimated 50\% of infections are produced by asymptomatic ---and thus, undetected--- cases \cite[]{Johansson2021}. Reducing the perceived cost of preventing onward SARS-CoV-2 transmission may thus include facilitating testing. For instance, by establishing mass testing programs or by offering tests through salivar samples instead of nasopharyngeal swabs \cite[]{Yee2021}, as well as at-home alternatives \cite[]{ValentineGraves2020}. 
%

%%
%All four studies used SEIR-type compartimental model for disease transmission, including symptomatic and asymptomatic individuals. 
%%
%\cite{Gupta2020} published a working paper where compliance to social distancing interventions was studied via mobility data and in terms of the production and consumption of commodities. They defined the utility function explicitly over regular commodities, occupation and health status. The authors found that information-based interventions had a large impact on mobility during the early stages of the epidemic, and that late adoption of state closure may induce lesser compliance of social-distance adoption and mobility reduction.
%%
%\cite{Kabir2020} studied the compliance of SAH interventions by including quarantine after exposure to the virus into the transmission model, as well as hospitalizations and immunity after recovery. The payoff depended on the relative costs perceived by individuals and the number of infected individuals. The authors found that individuals' compliance for SAH interventions wanes with time. In addition, they studied the final size of the epidemic and the fraction of hospitalized population in terms of the perceived cost and the transmission rate, concluding that SAH and natural immunity complement each other in controlling the epidemic and thus, hospital occupation, as long as cost is perceived low. 
%%
%\citet{Choi2020} accounted for imperfect vaccination and social distancing (reduction of the number of contacts, by a constant factor) in their transmission model. They analyzed the individuals' expected payoffs (which depended on the strategies' adoption rates and relative perceived costs) when individuals may adopt the strategies of social distancing, vaccination, or both, to avoid infection, finding the threshold relative costs (of vaccination and social distancing) that determined the selection over one strategy over the other. 
%%
%\cite{Jentsch2021} studied vaccination prioritizing, considering three strategies: i) targeting older first, ii) targeting younger first, iii) vaccinating the whole population uniformly, and iv) a contact-based strategy. The authors used an age-structured compartmental model accounting for imperfect vaccination and contacts by age and location. The decision model concerned the individual adherence to NPI. They calibrated their model using and using mobility data. Their main result in the best strategy (determined by the reduction in the number of deaths), depending on the reported cases and the proportion of vaccinated individuals.
